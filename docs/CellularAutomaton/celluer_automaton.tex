\documentclass[a4j,12pt]{jreport}
\title{個人プロジェクト}
\date{\today}
\begin{document}
\maketitle

\chapter{概要}
セルオートマトンの面白そうなトピックを真似してみる

\chapter{仕様}
背景とかを文章で書いていく。

\chapter{設計}
\section{ユースケース}
\section{概念モデリング}

\appendix

\section{FFT}

Complex conjugate-symmetric (CCS) vectorを次のように定義する。
整数$j\ (0\leq j \leq n-1)$に対して、次の式を計算する。
\begin{eqnarray}
    z_j &=& \sum_{k=0}^{n-1} t_k w^{-j k},\\\
    t_k &=& {\rm cmplx}(r_k,0),\\
    w &=& \exp\left(\frac{2\pi}{n}i\right)
\end{eqnarray}
ここで、$z_j$は次の性質を持つ。
\begin{eqnarray}
    z_{n/2 + j} = z_{n/2-j}^*
\end{eqnarray}
実際、
\begin{eqnarray}
    z_{n/2-j}^* &=& \sum_{k=0}^{n-1}t_k (w^{-(n/2-j)k})^*,
\end{eqnarray}
\begin{eqnarray}
    (w^{-(n/2-j)k})^* = \exp\left(-2^\pi\frac{(n/2-j)k}{n}i\right)^*
    = \exp\left(2^\pi\frac{(n/2-j)k}{n}i\right)
    = \exp\left(-2^\pi\frac{(n/2+j)k}{n}i+2\pi i\right)
    = \exp\left(-2^\pi\frac{(n/2+j)k}{n}i\right)
    = w^{(n/2+j)k}
\end{eqnarray}
となる。

複素数列$\{r_k\}$に対してForward FFTとInverse FFTは次のように定義される。
\begin{eqnarray}
    z_j &=& \sum_{k=0}^{n-1}r_k w^{-jk},\\
    r_j &=& \frac{1}{n}\sum_{k=0}^{n-1}z_k w^{jk}
\end{eqnarray}

\end{document}